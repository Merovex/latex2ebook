\setcounter{errorcontextlines}{400}

% line breaking tolerances
\tolerance=1000  % usually 200, infinite=10000
% last ditch effort: how much space OK to insert
%   in a line to avoid overfull
\setlength{\emergencystretch}{3em}  

% write metadata about screencasts to output file
\newwrite\screencastdata
\openout\screencastdata=\jobname.screencasts

\ifhtmloutput
%    \newcommand{\mtctitle}{}
\else
    \usepackage{minitoc}
    \renewcommand{\mtctitle}{}
    \usepackage[dvips,color,innerbars]{changebar}
\fi
\usepackage{xstring}  % provides StrSubstitute
\usepackage{natbib}
% natbib redefines the bibliography environment, so we have to override
% its definition after loading it to make the chapter bibs section-like
% rather than chapter-like
\makeatletter
  \renewenvironment{thebibliography}[1]
       {%\addtocounter{section}{1}
        %\section*{References In This Chapter}%
        \@mkboth{\MakeUppercase\bibname}{\MakeUppercase\bibname}%
        \list{\@biblabel{\@arabic\c@enumiv}}%
             {\settowidth\labelwidth{\@biblabel{#1}}%
              \leftmargin\labelwidth
              \advance\leftmargin\labelsep
              \@openbib@code
              \usecounter{enumiv}%
              \let\p@enumiv\@empty
              \renewcommand\theenumiv{\@arabic\c@enumiv}}%
        \sloppy
        \clubpenalty4000
        \@clubpenalty \clubpenalty
        \widowpenalty4000%
        \sfcode`\.\@m}
       {\def\@noitemerr
         {\@latex@warning{Empty `thebibliography' environment}}%
        \endlist}
\makeatother

\usepackage{bibunits}
\usepackage{ifthen}
\usepackage{tabularx}
\usepackage{array}
\usepackage{graphicx}
\usepackage{fancyvrb}
\usepackage{verbatim}
%\usepackage[font=small,labelfont=bf,margin=10pt]{caption}

% Textual ``figures''
\ifhtmloutput
  \newenvironment{textfigure}{}{}
\else
  \newenvironment{textfigure}{\hrule\small}{\vspace{0.5ex}\hrule}
\fi

% highlighting deltas from a previous explanation

\ifhtmloutput
  \newenvironment{changebar}{\HCode{<span class="changebar">}}{\HCode{</span>}}
\else
    \setlength{\changebarwidth}{0.5em}
    \definecolor{ChangebarColor}{rgb}{0.05,0,0.5} 
    \cbcolor{ChangebarColor}
\fi
\usepackage{endnotes}
\usepackage{listings} \usepackage{color}
\usepackage{textcomp} % needed for 'upquote' option below
\definecolor{ListingColor}{RGB}{248,248,248}
\lstset{ %
  language=Ruby,                % the language of the code
  upquote=true,                 % use straight quotes not curly quotes
  basicstyle=\scriptsize\ttfamily\upshape,       % the size and style of the fonts that are used for the code
  commentstyle=\slshape,
  xrightmargin=0.1in,
  xleftmargin=0.2in,
  numbers=left,                   % where to put the line-numbers
  numberstyle=\footnotesize,      % the size of the fonts that are used for the line-numbers
  stepnumber=1,                   % the step between two line-numbers. If it's 1, each line 
  % will be numbered
  numbersep=8pt,                  % how far the line-numbers are from the code
  backgroundcolor=\color{ListingColor},  % choose the background color. You must add \usepackage{color}
  showspaces=false,               % show spaces adding particular underscores
  showstringspaces=false,         % underline spaces within strings
  showtabs=false,                 % show tabs within strings adding particular underscores
  frame=trbL,                   % adds a frame around the code
  tabsize=2,                      % sets default tabsize to 2 spaces
  captionpos=b,                   % sets the caption-position to bottom
  breaklines=true,                % sets automatic line breaking
  breakatwhitespace=false,        % sets if automatic breaks should only happen at whitespace
  %title=\lstname,                 % show the filename of files included with \lstinputlisting;
  % also try caption instead of title
  escapeinside={\%*}{*)},         % if you want to add a comment within your code
  morekeywords={*,...}            % if you want to add more keywords to the set
}

\usepackage[figuresright]{rotating} 
\usepackage{floatpag} 
\rotfloatpagestyle{empty} 
\usepackage{makeidx}
  \makeindex
\usepackage{enumerate}
\usepackage{fancybox}

\usepackage{hyperref}
  \ifhtmloutput
    \hypersetup{hyperfootnotes=false}
  \fi


%\usepackage{glossaries}

% shortcuts
\newcommand{\fillinblank}{\_\_\_\_}
\newcommand{\n}{\tabularnewline}
\newcommand{\tg}{\textgreater}
\newcommand{\tl}{\textless}
\newcommand{\ttil}{\textasciitilde}
\newcommand{\tcaret}{\textasciicircum}
\ifhtmloutput
  \ifmobioutput
   \newcommand{\T}[1]{\HCode{<tt>}#1\HCode{</tt>}}
   \newcommand{\B}[1]{\HCode{<strong>}#1\HCode{</strong>}}
   \newcommand{\Sf}[1]{\HCode{<span class="sans-serif">}#1\HCode{</span>}}
   \newcommand{\C}[1]{\HCode{<span style="color: \#800;"><tt>}#1\HCode{</tt></span>}}
  \else
   \newcommand{\T}[1]{\HCode{<span class="typewriter">}#1\HCode{</span>}}
   \newcommand{\B}[1]{\HCode{<span class="strong">}#1\HCode{</span>}}
   \newcommand{\Sf}[1]{\HCode{<span class="sans-serif">}#1\HCode{</span>}}
   \newcommand{\C}[1]{\HCode{<span class="code-example">}#1\HCode{</span>}}
  \fi
\else
  \newcommand{\T}{\texttt}
  \newcommand{\B}{\textbf}
  \newcommand{\Sf}[1]{\textsf{\emph{#1}}}
  %\definecolor{TextHighlight}{rgb}{0.05,0,0.50}
  \definecolor{TextHighlight}{rgb}{0.1,0.1,0.1}
  \newcommand{\C}[1]{\textsf{\upshape{\textbf{#1}}}}
\fi

% make caption font smaller (tip from 
% http://dcwww.camd.dtu.dk/~schiotz/comp/LatexTips/LatexTips.html#captfont)
\newcommand{\captionfonts}{\scriptsize\bf}
\ifhtmloutput
 %
\else
  \makeatletter  % Allow the use of @ in command names
  \long\def\@makecaption#1#2{%
   \begin{centering}
    \vskip\abovecaptionskip % space between figure & caption
    \sbox\@tempboxa{{\captionfonts #1: #2}}%
    \ifdim \wd\@tempboxa >\hsize
      {\captionfonts #1: #2\par}
    \else
      \hbox to\hsize{\hfil\box\@tempboxa\hfil}%
    \fi
    \end{centering}
    \vskip\belowcaptionskip% space between caption & text
  }
  \makeatother   % Cancel the effect of \makeatletter
\fi

% hyperlinks

\ifhtmloutput
  \newcommand{\weblink}[2]{\href{#1}{#2}}
\else
  \newcommand{\weblink}[2]{#2\endnote{\expandafter \url{#1}}}
  %\newcommand{\weblink}[2]{#2 ({\url{#1}})}
\fi


% Quotation in small-ish type with attribution
% Example use:  
%  \makequotation{There's a sucker born every minute.}{W.C. Fields}

\ifhtmloutput
 \ifmobioutput
  \newcommand{\makequotation}[2]{\HCode{%
     <div class="makequotation" align="center">
      <table cellpadding="10" style="margin-left: 0;">
        <tr><td align="center" border="1" style="background-color: \#c0c0c0">}%
        #1\HCode{<i>} #2\HCode{</i>
        </td></tr>
      </table>
     </div>}}
 \else
  \newcommand{\makequotation}[2]{\begin{quote}\small\emph{#1} ---#2\end{quote}}
 \fi
\else
  \newcommand{\makequotation}[2]{\begin{quote}\small\emph{#1}

  \hfill---#2\end{quote}}
\fi

% Chapter front matter (used only by html build)
\newenvironment{chapterfrontmatter}{}{}

% Chapter overview (key messages)
% Example use:
%  \chapteroverview{User stories are important. Let's learn about them.}
%
%  implemented as Subsection for TOC inclusion:
\newcommand{\chapteroverview}[1]{
  \addtocontents{toc}{#1}
  \subsection*{#1}
  \vspace{0.25in}
  \ifhtmloutput\else\minitoc\fi
  \vfill\clearpage}

% Sidebar - appearance will vary depending on output target, and may not
% actually be along the side.
% Example use:
%   \begin{sidebar}{Historical note}
%     All this stuff will be in a sidebar.
%     You can include most elements that would go in normal text, like
%     figures, citations, etc.
%   \end{sidebar}

\definecolor{SidebarColor}{rgb}{0.9,0.9,0.9}
\ifhtmloutput
  \newenvironment{sidebar}[2][]%
    {\HCode{<span class="sidebar_name"><b><i>}#2{\HCode{</i></b></span>}}}
    {}
\else
  \newcommand{\thesidebarraise}{}
  \newenvironment{sidebar}[2][-2ex]%
    {\renewcommand{\thesidebarraise}{#1}%
     \begin{Sbox}\begin{minipage}{\marginparwidth}\sidebarfont\raggedright \textbf{#2}}%
    {\end{minipage}\end{Sbox}\marginpar{\vspace{\thesidebarraise}\colorbox{SidebarColor}{\TheSbox}}}
\fi

% Sidebars with photos - need to be small in PDF Output
% Small Kindle is 520x622 px; images 260x311 or smaller won't be resized on device
\ifhtmloutput
  \ifmobioutput
    \newenvironment{sidebargraphic}[3][]%
      {\HCode{<img width="150" src="#2_SB.jpg"/>}
       \HCode{<span class="sidebar_name"><b><i>}#3{\HCode{</i></b></span>}}}
      {}
  \else
    \newenvironment{sidebargraphic}[3][]%
      {\HCode{<img class="sidebarimage" src="#2.jpg">}
       \HCode{<span class="sidebar_name">}#3\HCode{</span>}}
      {}
  \fi
\else
  \newcommand{\thesidebarimg}{}% hack- numbered params can't appear in end content of \newenvironment
  \newenvironment{sidebargraphic}[3][0in]%
    {\renewcommand{\thesidebarraise}{#1}%
     \renewcommand{\thesidebarimg}{#2}%
     \begin{Sbox}\begin{minipage}{\marginparwidth}\sidebarfont\raggedright \textbf{#3}}%
    {\hfill\\ \includegraphics[width=\marginparwidth]{\thesidebarimg}\end{minipage}\end{Sbox}\marginpar{\vspace{\thesidebarraise}\colorbox{SidebarColor}{\TheSbox}}}
\fi



% Endnotes - in print version, footnotes/endnotes come at end of
% chapters

\newenvironment{tolearnmore}{\section{To Learn More}}{%
  % per-chapter bibliography
  \putbib
  % Endnotes
  \ifhtmloutput\else%
    \theendnotes
    \setcounter{endnote}{0}
  \fi
}

% Code - uses \verbatiminput to put in file 'undecorated' for HTML use,
% or \lstlisting to decorate it with styling and line numbers for PDF
\ifhtmloutput
 \ifmobioutput
  \newcommand{\codefile}[2][u/saasbook]{\HCode{<a class="pastebin" href="http://pastebin.com/#1">http://pastebin.com/#1</a><br/><pre class="code">}

    \begin{minipage}{\textwidth}% prevent gratuitous blank lines
    \verbatiminput{#2}%
    \end{minipage}%
    \HCode{</pre><hr>}}
 \else
  \newcommand{\codefile}[2][u/saasbook]{\HCode{<a class="pastebin" href="http://pastebin.com/#1">http://pastebin.com/#1</a><pre class="code">}
    \verbatiminput{#2}

    \HCode{</pre>}}
 \fi 
\else
  \newcommand{\codefile}[2][u/saasbook]{%
    \noindent\begin{minipage}{\textwidth}%
    \Sf{\scriptsize{}http://pastebin.com/#1}%
    \vspace{-1ex}%
    \lstinputlisting{#2}%
    \end{minipage}}
\fi

% Fallacies and pitfalls
%  \begin{fallacy}{The moon is made of green cheese.}
%    It really isn't.
%  \end{fallacy}

\ifhtmloutput
  \newcommand{\fallaciesandpitfalls}{}
\else
  \newcommand{\fallaciesandpitfalls}{%
  ~\vspace{-1.5ex}

  }
\fi

\ifhtmloutput
 \ifmobioutput
  \newenvironment{fallacy}[1]%
    {\HCode{<img src="icons/mobi/fallacy.png"><span style="font-weight: bold; color: red;">Fallacy: }#1\HCode{</span> }}
    {}
  \newenvironment{pitfall}[1]%
    {\HCode{<img src="icons/mobi/pitfall.png"><span style="font-weight: bold; color: red;">Pitfall: }#1\HCode{</span> }}
    {}
 \else
  \newenvironment{fallacy}[1]{\Picture{icons/fallacy.png class="marginicon"}\B{\emph{Fallacy:} #1}}{}
  \newenvironment{pitfall}[1]{\Picture{icons/pitfall.png class="marginicon"}\B{\emph{Pitfall:} #1}}{}
 \fi
\else
  \newcommand{\iconwithtext}[3]{%
     \vspace{1ex}%
     \hspace{-\parindent}%
     \begin{tabular}{b{0.25in}b{4.5in}}%
       \includegraphics[width=0.25in]{icons/#1} & \B{\emph{#2: }\/#3} \\ %
     \end{tabular}%
     \vspace{1ex}}
  \newenvironment{fallacy}[1]%
    {\iconwithtext{fallacy.pdf}{Fallacy}{#1}

    }
    {\vspace{0.5ex}}
  \newenvironment{pitfall}[1]%
    {\iconwithtext{pitfall.pdf}{Pitfall}{#1}

    }
    {\vspace{0.5ex}}
\fi
 
% Pull in Exercises
\newcommand{\exercises}[1]{\input{ch_#1/exercises_#1}}
                       
% Screencast - in print version, just a URL link; in HTML5 version,
% embedded quicktime player or youtube player??

\newcounter{screencast}[section]
\renewcommand*\thescreencast{\thesection.\arabic{screencast}}
\newcommand{\screencastprefix}{http://vimeo.com/}
\newcommand{\screencastsuffix}{}

\newcommand{\screencast}[5][saasbook/videos]{%
  \refstepcounter{screencast}
  \label{#2}\relax%
  % write metadata to output file
  \write\screencastdata{\thescreencast: #2@@#3@@#4##@5}
  %
  \ifhtmloutput
     \ifmobioutput
        \HCode{<hr>
               <div class="screencast"> Screencast \thescreencast: <a class="screencast-url" %
               href="\screencastprefix#1\screencastsuffix">#3</a>}%
        \HCode{<div class="screencastsummary">}#5\HCode{</div></div>
               <hr>}
     \else
         \HCode{<div class="screencast"><h2>Screencast \thescreencast: <span class="screencast_title">}#3\HCode{</span></h2>}
         \HCode{ <video src="#4" width="600" controls="controls">}
         \HCode{   <!-- for browsers that don't support embedded video tag -->}
         \HCode{   <a class="screencast-url" href="\screencastprefix#1\screencastsuffix">Watch on Vimeo</a>}
         \HCode{ </video>}
         \HCode{<div class="screencastsummary">}#5\HCode{</div></div>}
     \fi
  \else
    \hfill
     \begin{minipage}{0.9\textwidth}
       \rule{\textwidth}{0.5pt}\hfill\\
       \small\textbf{Screencast \thescreencast: #3.} \\
       \url{\screencastprefix#1\screencastsuffix}\\
       #5\relax \\
       \rule{\textwidth}{0.5pt}\hfill\\
     \end{minipage}
    \hfill
  \fi
}

% Elaboration - optional material that's longer than a sidebar

%\definecolor{ElaborationColor}{rgb}{0.05,0.0,0.50}
\definecolor{ElaborationColor}{rgb}{0.1,0.1,0.1}
\ifhtmloutput
 \ifmobioutput
  \newenvironment{elaboration}[1]%
    {\HCode{<hr><b>E</b><small><b>LABORATION:</b> #1<br/> }}
    {\HCode{</small><hr>}}
 \else
  \newenvironment{elaboration}[1]%
    {\HCode{<span class="elaboration_name">#1</span> }}
    {}
 \fi
\else
  \newcommand{\aboveelaborationskip}{1.5ex}
  \newcommand{\belowelaborationskip}{1.5ex}
  \newcommand{\elaborationpadding}{0.5ex}
  \newenvironment{elaboration}[1]%
    {\color{ElaborationColor} %
      \begin{Sbox}\begin{minipage}[r]{0.9\textwidth}%
          \vspace{\aboveelaborationskip}
          \rule{\textwidth}{2pt}
          \vspace{\elaborationpadding}
          \rule{1ex}{1ex}\hspace{2pt}\small\emph{\textbf{Elaboration: #1\relax}}
          \vspace{\elaborationpadding}
          \hrule
          \vspace{\elaborationpadding}
    }%
    {\vspace{\elaborationpadding}
     \hrule
     \end{minipage}\end{Sbox}
     \hfill\mbox{\TheSbox}
     \vspace{\belowelaborationskip}
    }
\fi

% LaTeX macros that aren't automatically translated to HTML
\ifhtmloutput
  \renewcommand{\copyright}{\HCode{&copy;}}
\fi

% forced page breaks
\ifhtmloutput
  \newcommand{\flushpage}{\clearpage\HCode{<mbp:pagebreak/>}}
\else
  \newcommand{\flushpage}{\clearpage}
\fi

% graphics - for HTML/mobi, use the gif version
\ifhtmloutput
  \newcommand{\picfigure}[4][width=\textwidth]{%
    \begin{figure}
     \Picture{#2.gif id="#3"}
       \ifthenelse{\equal{#4}{}}{}{\caption{\label{#3}#4}}
    \end{figure}}   
\else
  \newcommand{\picfigure}[4][width=\textwidth]{%
    \begin{figure}
     \includegraphics[#1]{#2}
     \ifthenelse{\equal{#4}{}}{}{\caption{\label{#3}#4}}
    \end{figure}}
\fi

% table figures - will eventually do something different for mobi
\newcommand{\tablefigure}[3]{%
  \begin{figure}
    \input{#1}
    \caption{\label{#2}%
#3}
  \end{figure}
}


% Summary - key points summarizing a chapter or section

\ifhtmloutput
  \ifmobioutput
    \newenvironment{summary}{}{}
  \else
    \newenvironment{summary}%
      {\HCode{<div class="summary">}}%
      {\HCode{</div>}}
  \fi
\else
  \renewcommand{\shadowsize}{2pt}
  \newenvironment{summary}%
    {\vspace{2ex}\begin{Sbox}\begin{minipage}{0.96\textwidth}\noindent}
    {\vspace{0.25\baselineskip}\end{minipage}\end{Sbox}\noindent\hspace{0in}\shadowbox{\TheSbox}}
\fi
   
% Note to self
\newcommand{\note}[1]{\fbox{#1}}

% DRYing out codefile examples
\newcommand{\codefilefigure}[4][u/saasbook]{%  filename, reflabel, caption
  \begin{figure}
  \codefile[#1]{#2}
  \caption{\label{#3}
   #4
  }
  \end{figure}%
}

% Crosscutting themes - icon or note in margin
\ifhtmloutput
 \ifmobioutput
  \newcommand{\crosscutting}[2]{%
    \HCode{<img align="right" width="40" src="icons/mobi/#2.png"/>&nbsp;}}
 \else
  \newcommand{\crosscutting}[2]{%
    \HCode{<img class="crosscutting" src="icons/#2.png"/>&nbsp;}}
 \fi
\else
  \newcommand{\crosscutting}[2]{%
    \marginpar{\vspace{#1}\includegraphics[width=0.5in]{icons/#2.png}}}
\fi
\newcommand{\dry}[1][0.5ex]{\crosscutting{#1}{dry}}
\newcommand{\reuse}[1][0.5ex]{\crosscutting{#1}{reuse}}
\newcommand{\codegen}[1][0.5ex]{\crosscutting{#1}{codegen}}
\newcommand{\concise}[1][0.5ex]{\crosscutting{#1}{concise}}
\newcommand{\coc}[1][0.5ex]{\crosscutting{#1}{coc}}
\newcommand{\legacy}[1][0.5ex]{\crosscutting{#1}{legacy}}
\newcommand{\beauty}[1][0.5ex]{\crosscutting{#1}{beauty}}
\newcommand{\tool}[1][0.5ex]{\crosscutting{#1}{tool}}
\newcommand{\learnbydoing}[1][0.5ex]{\crosscutting{#1}{learnbydoing}}
\newcommand{\automation}[1][0.5ex]{\crosscutting{#1}{automation}}

% ``Learn by doing''

% Table of contents formatting
\setcounter{tocdepth}{5}
%\newcommand{\chapteroverview}[1]{\addcontentsline{toc}{section}{#1}}

\newcommand{\gloss[1]}{\textbf{#1}}

% Exercises
\newtheorem{exercise}{Exercise}[chapter]
% Check Yourself
\newtheorem{checkyourself}{Self-Check}[section]
% answers embedded in self-checks
\ifhtmloutput
  \ifmobioutput
    \newenvironment{answer}{\HCode{<img height="15" width="15" src="icons/mobi/green_dot.gif"/>}}{}
  \else
    \newenvironment{answer}{}{}
  \fi    
\else
  \newenvironment{answer}%
    {\hfill \\ \noindent\upshape$\diamond$ }%
    {\rule{1ex}{1ex}~~}
\fi

% answers to exercises: omit from book, collect elsewhere
%\newenvironment{exanswer}{}{}
\let\exanswer=\comment
\let\endexanswer=\endcomment


% Wikipedia link

\newcommand{\w}[2][]{%
  \ifhtmloutput
    \ifthenelse{\equal{#1}{}}{%
      \StrSubstitute{#2}{ }{_}[\thewikilink]%
      \weblink{http://en.wikipedia.org/wiki/\thewikilink}{\emph{\B{#2}}}%
    }{%
      \StrSubstitute{#1}{ }{_}[\thewikilink]%
      \weblink{http://en.wikipedia.org/wiki/\thewikilink}{\emph{\B{#2}}}%
    }%
  \else%
    \textsf{\slshape\B{#2}}%
  \fi%
  \ifhtmloutput%
  \else%
    \index{#2}%
    \ifthenelse{\equal{#1}{}}{}{\index{#1}}%
    \relax%
  \fi%
}


% important term - not a link
\newcommand{\x}[1]{\emph{\B{#1}}}
